% TODO: Add custom LaTeX header directives here
\usepackage{xcolor}
\usepackage{scrlayer-scrpage}


\setlength{\parindent}{1.25cm} % sangria
\renewcommand{\baselinestretch}{1.2} % interlineado
\setlength{\parskip}{1em} % espacio entre parrafos

\definecolor{verdeUdeA}{RGB}{83,129,53} % color de la UdeA

\KOMAoptions{
  headsepline=1pt, % grosor linea en head
  automark,
  autooneside=false
}
\pagestyle{scrheadings} % necesario para poner linea en head
\setkomafont{headsepline}{\color{verdeUdeA}} % color linea en head

\clearpairofpagestyles % Necesario para poner el número en posición
\ohead{\pagemark} % número de página en esquina superior derecha
\ihead{\leftmark} % título del capítulo en esquina superior izquierda

% configuración de diferentes niveles de los headers

\renewcommand\raggedchapter{\centering}
\RedeclareSectionCommand[
  beforeskip=1em,
  afterskip=1em,
  font=\normalfont\normalsize\bfseries
]{chapter}

\RedeclareSectionCommand[
  beforeskip=1em,
  afterskip=1em,
  font=\normalfont\normalsize\bfseries
]{section}

\RedeclareSectionCommand[
  beforeskip=1em,
  afterskip=1em,
  font=\normalfont\normalsize\bfseries\itshape
]{subsection}

\RedeclareSectionCommand[
  beforeskip=1em,
  afterskip=1em,
  font=\normalfont\normalsize\bfseries,
  runin=true,
  indent=1.3cm
]{subsubsection}

\RedeclareSectionCommand[
  beforeskip=1em,
  afterskip=1em,
  font=\normalfont\normalsize\bfseries\itshape,
  runin=true,
  indent=1.3cm
]{paragraph}

% agrega el punto al final de los header de 4 y 5 nivel.

\makeatletter
\renewcommand{\sectionlinesformat}[4]{%
  \Ifstr{#1}{chapter}{\centering}{}% center section titles
  \@hangfrom{\hskip #2#3}{#4}%
}
\makeatother
\renewcommand{\sectioncatchphraseformat}[4]{%
  \hskip #2#3#4%
  \Ifstr{#1}{subsubsection}{.}{}% dot after subsection titles
  \Ifstr{#1}{paragraph}{.}{}
}

% Define el formato de las leyendas y descripciones de las figuras y tablas

\KOMAoption{captions}{nooneline}
\setkomafont{caption}{\itshape}
\setkomafont{captionlabel}{\normalfont\bfseries}
\setcapindent*{0pt}
\renewcommand*{\captionformat}{}

% Define el símbolo para el primer nivel de la lista de viñetas
\usepackage{enumitem}
\setlist[itemize,1]{label={\textbullet}, leftmargin=*}

\usepackage{hhline, colortbl}

%-------------------------------------------------------------------------------
%	TOC, LOT, Y LOF
%-------------------------------------------------------------------------------

% configuración del TOC
% hace que haya un interlineado entre cada item.

\DeclareTOCStyleEntry[
  entryformat=\normalfont\normalsize,
  pagenumberformat={}
]{tocline}{part}

\DeclareTOCStyleEntry[
  entryformat=\normalfont\normalsize,
  pagenumberformat={}
]{tocline}{chapter}

\DeclareTOCStyleEntry[
  beforeskip=1em
]{tocline}{section}

\DeclareTOCStyleEntry[
beforeskip=1em
]{tocline}{subsection}

\DeclareTOCStyleEntry[
beforeskip=1em
]{tocline}{subsubsection}

\DeclareTOCStyleEntry[
beforeskip=1em
]{tocline}{paragraph}