\markright{\mititulo \hspace{5cm} \thepage}
\pagestyle{fancy}
\section*{Resumen}
\addcontentsline{toc}{section}{Resumen}%-------todo parece bien pero esta linea genera un error desconocido 

\noindent El resumen permite identificar la esencia del escrito, mencionando brevemente el objetivo y la metodología, así como los resultados y las conclusiones (mínimo 150, máximo 250 palabras).

\vspace{1cm}\textit{Palabras clave:} artículo científico, artículo de revisión, investigación, estilos de citación


\newpage
%----------------------------------

\section*{Abstract}
\addcontentsline{toc}{section}{Abstract}%-------todo parece bien pero esta linea genera un error desconocido 

\noindent El abstract es el mismo resumen pero en idioma inglés. Conserva la misma extensión o aproximada, es decir, mínimo 150 y máximo 250 palabras.



\vspace{1cm}\textit{Keywords:} scientific article, review article, research, citation styles

\newpage
%----------------------------------


\section*{Introducción}
\addcontentsline{toc}{section}{Introducción}%-------todo parece bien pero esta linea genera un error desconocido 

En la introducción se menciona claramente el para qué y el porqué del documento, se incluye el planteamiento del problema, el objetivo, preguntas de investigación, la justificación.

Si bien se prefiere la narración en tercera persona (se realizaron las encuestas, se publicaron resultados, se establecieron parámetros, etc.), en Normas APA también se aprueba el uso de primera persona singular para un solo autor (realicé las encuestas) o primera persona plural (o mayestático) para dos o más autores (realizamos las encuestas); en todo caso, consulta con tu asesor el estilo a adoptar en la investigación\footnote{No utilices los pies de página para citas bibliográficas. Los pies de página se utilizan para complementar información del texto, procura que sean fragmentos cortos para no distraer o confundir al lector.}.

No menos importante es la utilización de conectores que unen elementos de una oración, tener una buena variedad de estos enriquecen la estructura y redacción del texto. Algunos ejemplos:

\begin{tabular}{ll}
    Sin embargo         & En conclusión \\
    Puesto que          & En pocas palabras \\
    Por consiguiente    & A continuación \\
    Dado que            & Acto seguido \\
    Teniendo en cuenta  & Con motivo de \\
    Entonces            & A saber \\
    Simultáneamente     & De la misma forma \\
    Posiblemente        & En síntesis \\
    En efecto           & Así \\
    Ya que              & Para concluir \\
    Ahora bien          & Luego \\
    En cambio           & Resumiendo \\
    En cuanto a         & De igual manera \\
    El siguiente punto es& Al mismo tiempo \\
    Así pues            & Probablemente \\
    Recapitulando       & Indiscutiblemente 
\end{tabular}

\newpage